\documentclass{article}

\usepackage{tabularx}
\usepackage{booktabs}

\title{Problem Statement and Goals\\LiveNeuro}

\author{\ Bo Liang}

\date{January 20th 2025}

%% Comments

\usepackage{color}

\newif\ifcomments\commentstrue %displays comments
%\newif\ifcomments\commentsfalse %so that comments do not display

\ifcomments
\newcommand{\authornote}[3]{\textcolor{#1}{[#3 ---#2]}}
\newcommand{\todo}[1]{\textcolor{red}{[TODO: #1]}}
\else
\newcommand{\authornote}[3]{}
\newcommand{\todo}[1]{}
\fi

\newcommand{\wss}[1]{\authornote{blue}{SS}{#1}} 
\newcommand{\plt}[1]{\authornote{magenta}{TPLT}{#1}} %For explanation of the template
\newcommand{\an}[1]{\authornote{cyan}{Author}{#1}}

%% Common Parts

\newcommand{\progname}{Live Neuro} % PUT YOUR PROGRAM NAME HERE
\newcommand{\authname}{Bo Liang} % AUTHOR NAMES

\usepackage{hyperref}
    \hypersetup{colorlinks=true, linkcolor=blue, citecolor=blue, filecolor=blue,
                urlcolor=blue, unicode=false}
    \urlstyle{same}
                                


\begin{document}

\maketitle

\begin{table}[hp]
\caption{Revision History} \label{TblRevisionHistory}
\begin{tabularx}{\textwidth}{llX}
\toprule
\textbf{Date} & \textbf{Developer(s)} & \textbf{Change}\\
\midrule
January 20 2025 & Bo Liang & Initial Draft\\
\bottomrule
\end{tabularx}
\end{table}

\section{Problem Statement}

Neuron data based on human brain activity is often complex and multi-sourced. To clearly represent such data, it is common to use multiple data plots for visualization. However, due to the differing coordinate dimensions of various data sets, it becomes challenging for users to track specific data points across multiple plots at a given time and to comprehend the complete data transformation process. This hinders the understanding of the overall process.

\subsection{Problem}
This project aims to develop an interactive data visualization method that links multiple data plots together.
\subsection{Inputs and Outputs}
\subsubsection{input}
Neuron data, electrical stimulation signals, MRI data.
\subsubsection{output}
Interactive data visualization plots for users.

\subsection{Stakeholders}
Stakeholders include researchers, scholars, and students who aim to gain a clear understanding of the complete neuron stimulation process.
\subsection{Environment}
The project will be developed in Python 3.11 on Windows,macOS and Linux.


\section{Goals}
\begin{enumerate}
    \item Enable user interaction with data plots, allowing users to freely view specific data at any position on the plots.
    \item 2. Implement linkage between different data plots and support user interaction.
\end{enumerate}
\section{Stretch Goals}
Allow flexible configuration of different types of data, enabling customizable and dynamic linkages.
\section{Challenge Level and Extras}

This project poses a general level of challenge. While it is not a research-oriented project, it requires a certain level of understanding of neuroscience-related data models, extensive software development experience, and the ability to iterate based on user feedback and habits to ensure an optimal user experience.


\end{document}