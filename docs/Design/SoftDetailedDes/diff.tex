\documentclass[12pt, titlepage]{article}
%DIF LATEXDIFF DIFFERENCE FILE



\usepackage{amsmath, mathtools}

\usepackage[round]{natbib}
\usepackage{amsfonts}
\usepackage{amssymb}
\usepackage{graphicx}
\usepackage{colortbl}
\usepackage{xr}
\usepackage{hyperref}
\usepackage{longtable}
\usepackage{xfrac}
\usepackage{tabularx}
\usepackage{float}
\usepackage{siunitx}
\usepackage{booktabs}
\usepackage{multirow}
\usepackage[section]{placeins}
\usepackage{caption}
\usepackage{fullpage}

\hypersetup{
bookmarks=true,     % show bookmarks bar?
colorlinks=true,       % false: boxed links; true: colored links
linkcolor=red,          % color of internal links (change box color with linkbordercolor)
citecolor=blue,      % color of links to bibliography
filecolor=magenta,  % color of file links
urlcolor=cyan          % color of external links
}

\usepackage{array}

\externaldocument{../../SRS/SRS}

\input{../../Comments}
%% Common Parts

\newcommand{\progname}{Live Neuro} % PUT YOUR PROGRAM NAME HERE
\newcommand{\authname}{Bo Liang} % AUTHOR NAMES

\usepackage{hyperref}
    \hypersetup{colorlinks=true, linkcolor=blue, citecolor=blue, filecolor=blue,
                urlcolor=blue, unicode=false}
    \urlstyle{same}
                                

%DIF PREAMBLE EXTENSION ADDED BY LATEXDIFF
%DIF UNDERLINE PREAMBLE %DIF PREAMBLE
\RequirePackage[normalem]{ulem} %DIF PREAMBLE
\RequirePackage{color}\definecolor{RED}{rgb}{1,0,0}\definecolor{BLUE}{rgb}{0,0,1} %DIF PREAMBLE
\providecommand{\DIFaddtex}[1]{{\protect\color{blue}\uwave{#1}}} %DIF PREAMBLE
\providecommand{\DIFdeltex}[1]{{\protect\color{red}\sout{#1}}}                      %DIF PREAMBLE
%DIF SAFE PREAMBLE %DIF PREAMBLE
\providecommand{\DIFaddbegin}{} %DIF PREAMBLE
\providecommand{\DIFaddend}{} %DIF PREAMBLE
\providecommand{\DIFdelbegin}{} %DIF PREAMBLE
\providecommand{\DIFdelend}{} %DIF PREAMBLE
\providecommand{\DIFmodbegin}{} %DIF PREAMBLE
\providecommand{\DIFmodend}{} %DIF PREAMBLE
%DIF FLOATSAFE PREAMBLE %DIF PREAMBLE
\providecommand{\DIFaddFL}[1]{\DIFadd{#1}} %DIF PREAMBLE
\providecommand{\DIFdelFL}[1]{\DIFdel{#1}} %DIF PREAMBLE
\providecommand{\DIFaddbeginFL}{} %DIF PREAMBLE
\providecommand{\DIFaddendFL}{} %DIF PREAMBLE
\providecommand{\DIFdelbeginFL}{} %DIF PREAMBLE
\providecommand{\DIFdelendFL}{} %DIF PREAMBLE
%DIF HYPERREF PREAMBLE %DIF PREAMBLE
\providecommand{\DIFadd}[1]{\texorpdfstring{\DIFaddtex{#1}}{#1}} %DIF PREAMBLE
\providecommand{\DIFdel}[1]{\texorpdfstring{\DIFdeltex{#1}}{}} %DIF PREAMBLE
\newcommand{\DIFscaledelfig}{0.5}
%DIF HIGHLIGHTGRAPHICS PREAMBLE %DIF PREAMBLE
\RequirePackage{settobox} %DIF PREAMBLE
\RequirePackage{letltxmacro} %DIF PREAMBLE
\newsavebox{\DIFdelgraphicsbox} %DIF PREAMBLE
\newlength{\DIFdelgraphicswidth} %DIF PREAMBLE
\newlength{\DIFdelgraphicsheight} %DIF PREAMBLE
% store original definition of \includegraphics %DIF PREAMBLE
\LetLtxMacro{\DIFOincludegraphics}{\includegraphics} %DIF PREAMBLE
\newcommand{\DIFaddincludegraphics}[2][]{{\color{blue}\fbox{\DIFOincludegraphics[#1]{#2}}}} %DIF PREAMBLE
\newcommand{\DIFdelincludegraphics}[2][]{% %DIF PREAMBLE
\sbox{\DIFdelgraphicsbox}{\DIFOincludegraphics[#1]{#2}}% %DIF PREAMBLE
\settoboxwidth{\DIFdelgraphicswidth}{\DIFdelgraphicsbox} %DIF PREAMBLE
\settoboxtotalheight{\DIFdelgraphicsheight}{\DIFdelgraphicsbox} %DIF PREAMBLE
\scalebox{\DIFscaledelfig}{% %DIF PREAMBLE
\parbox[b]{\DIFdelgraphicswidth}{\usebox{\DIFdelgraphicsbox}\\[-\baselineskip] \rule{\DIFdelgraphicswidth}{0em}}\llap{\resizebox{\DIFdelgraphicswidth}{\DIFdelgraphicsheight}{% %DIF PREAMBLE
\setlength{\unitlength}{\DIFdelgraphicswidth}% %DIF PREAMBLE
\begin{picture}(1,1)% %DIF PREAMBLE
\thicklines\linethickness{2pt} %DIF PREAMBLE
{\color[rgb]{1,0,0}\put(0,0){\framebox(1,1){}}}% %DIF PREAMBLE
{\color[rgb]{1,0,0}\put(0,0){\line( 1,1){1}}}% %DIF PREAMBLE
{\color[rgb]{1,0,0}\put(0,1){\line(1,-1){1}}}% %DIF PREAMBLE
\end{picture}% %DIF PREAMBLE
}\hspace*{3pt}}} %DIF PREAMBLE
} %DIF PREAMBLE
\LetLtxMacro{\DIFOaddbegin}{\DIFaddbegin} %DIF PREAMBLE
\LetLtxMacro{\DIFOaddend}{\DIFaddend} %DIF PREAMBLE
\LetLtxMacro{\DIFOdelbegin}{\DIFdelbegin} %DIF PREAMBLE
\LetLtxMacro{\DIFOdelend}{\DIFdelend} %DIF PREAMBLE
\DeclareRobustCommand{\DIFaddbegin}{\DIFOaddbegin \let\includegraphics\DIFaddincludegraphics} %DIF PREAMBLE
\DeclareRobustCommand{\DIFaddend}{\DIFOaddend \let\includegraphics\DIFOincludegraphics} %DIF PREAMBLE
\DeclareRobustCommand{\DIFdelbegin}{\DIFOdelbegin \let\includegraphics\DIFdelincludegraphics} %DIF PREAMBLE
\DeclareRobustCommand{\DIFdelend}{\DIFOaddend \let\includegraphics\DIFOincludegraphics} %DIF PREAMBLE
\LetLtxMacro{\DIFOaddbeginFL}{\DIFaddbeginFL} %DIF PREAMBLE
\LetLtxMacro{\DIFOaddendFL}{\DIFaddendFL} %DIF PREAMBLE
\LetLtxMacro{\DIFOdelbeginFL}{\DIFdelbeginFL} %DIF PREAMBLE
\LetLtxMacro{\DIFOdelendFL}{\DIFdelendFL} %DIF PREAMBLE
\DeclareRobustCommand{\DIFaddbeginFL}{\DIFOaddbeginFL \let\includegraphics\DIFaddincludegraphics} %DIF PREAMBLE
\DeclareRobustCommand{\DIFaddendFL}{\DIFOaddendFL \let\includegraphics\DIFOincludegraphics} %DIF PREAMBLE
\DeclareRobustCommand{\DIFdelbeginFL}{\DIFOdelbeginFL \let\includegraphics\DIFdelincludegraphics} %DIF PREAMBLE
\DeclareRobustCommand{\DIFdelendFL}{\DIFOaddendFL \let\includegraphics\DIFOincludegraphics} %DIF PREAMBLE
%DIF END PREAMBLE EXTENSION ADDED BY LATEXDIFF

\begin{document}

\title{Module Interface Specification for \progname{}}

\author{\authname}

\date{\today}

\maketitle

\pagenumbering{roman}

\section{Revision History}

\begin{tabularx}{\textwidth}{p{3cm}p{2cm}X}
\toprule {\bf Date} & {\bf Version} & {\bf Notes}\\
\midrule
March 17 & 1.0 & Initial Draft\\
\DIFaddbegin \DIFadd{Apr 11 }& \DIFadd{1.1 }& \DIFadd{Modify}\\
\DIFaddend \bottomrule
\end{tabularx}

~\newpage

\section{Symbols, Abbreviations and Acronyms}

See SRS Documentation at \href{https://github.com/liang-bo96/CAS741/blob/main/docs/SRS/SRS.pdf}{SRS}


\newpage

\tableofcontents

\newpage

\pagenumbering{arabic}

\section{Introduction}

This document specifies the interfaces for modules in the Live Neuro system, an interactive neural data visualization tool. It complements the
\href{https://github.com/liang-bo96/CAS741/blob/main/docs/SRS/SRS.pdf}{SRS}
and \href{https://github.com/liang-bo96/CAS741/blob/main/docs/Design/SoftArchitecture/MG.pdf}{MG}, with full implementation details available at
\href{https://github.com/liang-bo96/CAS741}{ GitHub Repository}.

\section{Notation}


The following table summarizes the primitive data types used by \progname.

\begin{center}
\renewcommand{\arraystretch}{1.2}
\noindent
\begin{tabular}{l l p{7.5cm}}
\toprule
\textbf{Data Type} & \textbf{Notation} & \textbf{Description}\\
\midrule
String & $\mathbb{S}$ & a character string \\
\DIFdelbegin \DIFdel{data type dictionary }\DIFdelend \DIFaddbegin \DIFadd{Dict}[\DIFadd{str, Any}]
 \DIFaddend & $\mathbb{D}$ & a storage type for KV structures \\
bool & $\mathbb{B}$ &  Boolean data type, which can hold one of two values: either true or false \\
integer & $\mathbb{Z}$ & a number without a fractional component in (-$\infty$, $\infty$) \\
natural number & $\mathbb{N}$ & a number without a fractional component in [1, $\infty$) \\
real & $\mathbb{R}$ & any number in (-$\infty$, $\infty$)\\
\bottomrule
\end{tabular}
\end{center}

\noindent
The specification of \progname \ uses some derived data types: sequences, strings, and
tuples. Sequences are lists filled with elements of the same data type. Strings
are sequences of characters. Tuples contain a list of values, potentially of
different types. In addition, \progname \ uses functions, which
are defined by the data types of their inputs and outputs. Local functions are
described by giving their type signature followed by their specification.

\section{Module Decomposition}

The following table is taken directly from the Module Guide document for this project.

\begin{table}[h!]
\centering
\begin{tabular}{p{0.3\textwidth} p{0.5\textwidth}p{0.1\textwidth}}
\toprule
\textbf{Level 1} & \textbf{Level 2}& \textbf{Module ID}\\
\midrule

{Hardware-Hiding Module} &  Hardware-Hiding Module
 & M1 \\
\midrule

\multirow{3}{0.3\textwidth}{Behaviour-Hiding Module}
& Input Format Module & M2\\
& Data Processing Module & M3\\
& Visualization Module & M4\\

\midrule

\multirow{1}{0.3\textwidth}{Software Decision Module} & TRF Calculation Module & M5\\
\bottomrule

\end{tabular}
\caption{Module Hierarchy}
\label{TblMH}
\end{table}


\newpage
~\newpage

\section{MIS of Hardware-Hiding Module(M1)}

\subsection{Module}
M1: OS Abstraction Layer

\subsection{Uses}

\item Directly interacts with OS APIs (e.g., file I/O, hardware drivers).


\subsection{Syntax}

\subsubsection{Exported Access Programs}

\begin{center}
\begin{tabular}{p{2cm} p{4cm} p{4cm} p{4cm}}
\hline
\textbf{Name} & \textbf{In} & \textbf{Out} & \textbf{Exceptions} \\
\hline
readFile & $\mathbb{S}$ (file path) & Sequence & FileNotFoundError \\
savePlot & plot data & file(.jpeg) & SavePlotError \\
\hline
\end{tabular}
\end{center}

\subsection{Semantics}

\subsubsection{Environment Variables}

File \DIFdelbegin \DIFdel{system, display hardware}\DIFdelend \DIFaddbegin \DIFadd{system-fs, display hardware-dh}\DIFaddend .
\subsubsection{Assumptions}
OS compatibility (Linux/Windows/macOS)

\subsubsection{Access Routine Semantics}

\begin{itemize}
\item readFile(): Reads neural data from disk, returns a sequence of \DIFdelbegin \DIFdel{$\mathbb{R}$ numbers}\DIFdelend \DIFaddbegin \DIFadd{characters}\DIFaddend .


\item savePlot(): Renders visualization output to screen or file \DIFaddbegin \DIFadd{under .}\DIFaddend .\DIFaddbegin \DIFadd{/visualization/output.
}\DIFaddend


\end{itemize}

\DIFaddbegin \newpage


\DIFaddend \section{MIS of Input Format Module}

\subsection{Module}
Multi-Format MEG/EEG Data Parser

\subsection{Uses}

\item M1 (readFile for raw data loading).

\subsection{Syntax}

\subsubsection{Exported Constants}

supported formats = [EDF, FIF\DIFdelbegin \DIFdel{, BrainVision}\DIFdelend ] (supported data formats).\\
max channels = 256 (maximum allowed channels per dataset).


\subsubsection{Exported Access Programs}

\begin{center}
\begin{tabular}{p{2cm} p{4cm} p{4cm} p{2cm}}
\hline
\textbf{Name} & \textbf{In} & \textbf{Out} & \textbf{Exceptions} \\
\hline
load edf() & $\mathbb{S}$ (EDF file path) & $\mathbb{D}$<$\mathbb{S}$: $\mathbb{R}$[]> & EDFFileError \\
load fif() & $\mathbb{S}$ (FIF file path) & $\mathbb{D}$<$\mathbb{S}$: $\mathbb{R}$[]> & FIFFileError \\
load brainvision() & $\mathbb{S}$ (.vhdr file path) & $\mathbb{D}$<$\mathbb{S}$: $\mathbb{R}$[]> & VHDRFileError \\
\hline
\end{tabular}
\end{center}

\DIFaddbegin \textbf{\DIFadd{Notation Explanation:}}


\begin{itemize}


 \item \texttt{\DIFadd{$\mathbb{R}$}[]} \DIFadd{– a one-dimensional array of real numbers (e.g., one EEG channel)
}


\end{itemize}



\DIFaddend \subsection{Semantics}

\subsubsection{State Variables}

\begin{itemize}
\item currentFormat: $\mathbb{S}$(last detected data format, e.g., \DIFdelbegin \DIFdel{"FIF")}\DIFdelend \DIFaddbegin \DIFadd{FIF(enum type))}\DIFaddend .



\DIFaddbegin \begin{itemize}
\DIFaddend \item
\DIFdelbegin \DIFdel{metadataCache:
$\mathbb{D}$(cached metadata from parsed files ).
}\DIFdelend \DIFaddbegin \DIFadd{Stores the most recently detected data format.
}\DIFaddend


\DIFaddbegin \DIFadd{FormatType is an }\textbf{\DIFadd{enumerated type}} \DIFadd{with the following values:
}




\begin{itemize}
\item
\DIFadd{FIF
}



\item
\DIFadd{EDF
}



\DIFaddend \end{itemize}




\DIFaddbegin \end{itemize}





\end{itemize}


\DIFaddend \subsubsection{Environment Variables}




\DIFdelbegin \subsubsection{\DIFdel{Access Routine Semantics}}
%DIFAUXCMD
\addtocounter{subsubsection}{-1}%DIFAUXCMD
\DIFdelend \begin{itemize}
\item
\DIFdelbegin \DIFdel{load edf(}\DIFdelend \DIFaddbegin \textbf{\DIFadd{DATA PATH: String}}\textbf{}


\DIFadd{Path to the directory where data files are stored.
}


\DIFadd{This value can be passed explicitly or set using an environment variable (e.g., LIVE NEURO DATA PATH}\DIFaddend )\DIFaddbegin \DIFadd{.
}



\end{itemize}




\subsubsection{\DIFadd{Access Routine Semantics}}
\begin{itemize}




\subsubsection*{\textbf{\DIFadd{load\_edf(filepath: String)}}





\begin{itemize}
\item
\textbf{\DIFadd{Purpose}}\DIFadd{: Loads data from an EDF (European Data Format) file and returns EEG time-series data.
}



\item
\textbf{\DIFadd{Input}}\DIFadd{:
}




\begin{itemize}
\item
\DIFadd{filepath: Path to the .edf file as a string.
}



\end{itemize}




\item
\textbf{\DIFadd{Output}}\DIFaddend :
\DIFdelbegin %DIFDELCMD < \subitem %%%
\DIFdel{Output: }%DIFDELCMD < \subsubitem %%%
\DIFdel{data: Time-series array of shape}\DIFdelend \DIFaddbegin




\begin{itemize}
\item
\DIFadd{data: A data Dict}\DIFaddend [\DIFdelbegin \DIFdel{channels × samples}\DIFdelend \DIFaddbegin \DIFadd{String : array}\DIFaddend ]
\DIFdelbegin \DIFdel{.
    }%DIFDELCMD < \subitem %%%
\DIFdel{Exception:
}%DIFDELCMD < \subsubitem %%%
\DIFdel{EDFHeaderError: Invalid EDF header structure.
}%DIFDELCMD < \subitem %%%
\DIFdel{Implementation: Uses eelbrain for EDF parsing.}\DIFdelend \DIFaddbegin



\end{itemize}




\DIFaddend \item
\DIFdelbegin \DIFdel{load fif()
:
    }%DIFDELCMD < \subitem %%%
\DIFdel{Output:
        }%DIFDELCMD < \subsubitem %%%
\DIFdel{data: }\DIFdelend \DIFaddbegin \textbf{\DIFadd{Exceptions}}\DIFadd{:
}




\begin{itemize}
\item
\DIFadd{EDFFileError: Raised if the EDF file is malformed or unsupported.
}



\end{itemize}




\item
\textbf{\DIFadd{Implementation Detail}}\DIFadd{: Uses mne.io.read\_raw\_edf()
}

\end{itemize}









\subsubsection*{\textbf{\DIFadd{load\_fif(filepath: String) → mne.io.Raw}}}





\begin{itemize}
\item
\textbf{\DIFadd{Purpose}}\DIFadd{: Loads }\DIFaddend MEG/EEG sensor data \DIFaddbegin \DIFadd{from an MNE .fif file}\DIFaddend .
\DIFdelbegin %DIFDELCMD < \subitem %%%
\DIFdel{Exception:
        }%DIFDELCMD < \subsubitem %%%
\DIFdel{FIFFileError: FIF fileversion mismatch or missing data tags.
}%DIFDELCMD < \subitem %%%
\DIFdel{Implementation:
Relies on mne-python library.
  }\DIFdelend \DIFaddbegin



\DIFaddend \item
\DIFdelbegin \DIFdel{load brainvision(): }%DIFDELCMD < \subitem %%%
\DIFdel{Output:
        }%DIFDELCMD < \subsubitem %%%
\DIFdel{data: EEG data segmented by markers
    }%DIFDELCMD < \subitem %%%
\DIFdel{Exception:
        }%DIFDELCMD < \subsubitem %%%
\DIFdel{VHDRParseError: Inconsistent header fields in.vhdr file .
}\DIFdelend \DIFaddbegin \textbf{\DIFadd{Input}}\DIFadd{:
}




\begin{itemize}
\item
\DIFadd{filepath: Path to the .fif file as a string.
}



\DIFaddend \end{itemize}




\DIFdelbegin \subsubsection{\DIFdel{Local Functions}}
%DIFAUXCMD
\addtocounter{subsubsection}{-1}%DIFAUXCMD
\DIFdelend \DIFaddbegin \item
\textbf{\DIFadd{Output}}\DIFadd{:
}




\DIFaddend \begin{itemize}
\item
\DIFdelbegin \DIFdel{parse edf header(): Extracts EDF header fields and validates integrity.
}%DIFDELCMD < \\
%DIFDELCMD < %%%
\DIFdelend \DIFaddbegin \DIFadd{data: A Raw object from the mne library.
}



\end{itemize}




\DIFaddend \item
\DIFdelbegin \DIFdel{align brainvision files():
Synchronizes.vhdr, .vmrk, and.
eeg data.}\DIFdelend \DIFaddbegin \textbf{\DIFadd{Exceptions}}\DIFadd{:
}




\begin{itemize}
\item
\DIFadd{FIFFileError: Raised for version mismatch, file corruption, or missing tags.
}



\DIFaddend \end{itemize}




\DIFaddbegin \item
\textbf{\DIFadd{Implementation Detail}}\DIFadd{: Calls mne.io.read\_raw\_fif().
}



\end{itemize}




\end{itemize}




\DIFaddend \section{MIS of Data Processing Module}

\subsection{Module}

Statistical Preprocessing and Validation\\
Primary Function: Validates input data integrity and applies statistical preprocessing to neural signals.
\subsection{Uses}

\begin{itemize}
\item M2(Input Format Module): Receives parsed raw data (e.g., MEG/EEG time-series).


\item M5(TRF Calculation Module): Provides preprocessed data for dipole current computation.


\end{itemize}



\subsection{Syntax}


\subsubsection{Exported Access Programs}

\begin{center}
\begin{tabular}{p{2cm} p{4cm} p{4cm} p{2cm}}
\hline
\textbf{Name} & \textbf{In} & \textbf{Out} & \textbf{Exceptions} \\
\hline
validate input & $\mathbb{D}$<$\mathbb{S}$: $\mathbb{R}$[]> & $\mathbb{B}$ & InvalidDataError \\
compute statistics & $\mathbb{R}$[] & $\mathbb{D}$<$\mathbb{S}$: $\mathbb{R}$> & NaNError \\

\hline
\end{tabular}
\end{center}

\DIFaddbegin \DIFadd{Although input validation could technically be performed immediately after data is loaded, we intentionally separate the }\textbf{\DIFadd{validation logic}} \DIFadd{into its own module for reasons of }\textbf{\DIFadd{modularity}}\DIFadd{, }\textbf{\DIFadd{reusability}}\DIFadd{, and }\textbf{\DIFadd{maintainability}}\DIFadd{.
}


\DIFadd{Specifically:
}


\begin{itemize}
\item
\textbf{\DIFadd{Validation is not only for file loading}}\DIFadd{: In many use cases, data may come from }\textbf{\DIFadd{non-file sources}}\DIFadd{, such as the }\textbf{\DIFadd{internet}}\DIFadd{, }\textbf{\DIFadd{APIs}}\DIFadd{, or }\textbf{\DIFadd{in-memory structures}}\DIFadd{. Centralizing validation ensures consistency across all entry points.
}

\item
\textbf{\DIFadd{Validation rules evolve}}\DIFadd{: Scientific or clinical datasets may have }\textbf{\DIFadd{changing standards}} \DIFadd{or }\textbf{\DIFadd{conditional rules}}\DIFadd{. Keeping validation separate allows updates to these rules }\textbf{\DIFadd{without modifying the loader}} \DIFadd{or tightly coupled modules.
}

\item
\textbf{\DIFadd{Cleaner architecture}}\DIFadd{: Following the }\textbf{\DIFadd{Single Responsibility Principle}}\DIFadd{, the file-loading module focuses purely on parsing and decoding data formats (e.g., EDF, FIF), while the validation module handles semantic correctness and integrity.
}

\item
\textbf{\DIFadd{Testing and reuse}}\DIFadd{: A standalone validation module can be tested independently and reused in }\textbf{\DIFadd{preprocessing pipelines}}\DIFadd{, }\textbf{\DIFadd{live data streaming}}\DIFadd{, or even }\textbf{\DIFadd{user-uploaded data checks}}\DIFadd{.
}

\end{itemize}





\DIFadd{Therefore, validation is performed in a }\textbf{\DIFadd{later module}}\DIFadd{, after the raw data is loaded, to support broader use cases and maintain long-term flexibility of the system.
}

\DIFaddend \subsection{Semantics}

\subsubsection{State Variables}
\DIFaddbegin



\subsubsection*{\textbf{\DIFadd{validatedSignals}}}





\DIFaddend \begin{itemize}
\item
\DIFdelbegin \DIFdel{validatedSignals: $\mathbb{D}$<$\mathbb{S}$: $\mathbb{R}$}%DIFDELCMD < []%%%
\DIFdel{>(cached validated data)
.
}\DIFdelend \DIFaddbegin \textbf{\DIFadd{Initial State}}\DIFadd{: None
}\DIFaddend



\item
\DIFdelbegin \DIFdel{baselineStats:
$\mathbb{D}$<$\mathbb{S}$: $\mathbb{R}$>(mean and std of reference signals).
}\DIFdelend \DIFaddbegin \textbf{\DIFadd{Updated By}}\DIFadd{: validate\_input(data)
}\DIFaddend



\DIFaddbegin \item
\textbf{\DIFadd{Transition}}\DIFadd{:
}




\begin{itemize}
\item
\DIFadd{If data passes validation, validatedSignals ← data
}



\item
\DIFadd{If validation fails, validatedSignals remains unchanged
}



\DIFaddend \end{itemize}




\DIFaddbegin \end{itemize}








\subsubsection*{\textbf{\DIFadd{baselineStats}}}





\begin{itemize}
\item
\textbf{\DIFadd{Initial State}}\DIFadd{: None
}



\item
\textbf{\DIFadd{Updated By}}\DIFadd{: compute\_statistics(data)
}



\item
\textbf{\DIFadd{Transition}}\DIFadd{:
}




\begin{itemize}
\item
\DIFadd{Computes mean and standard deviation from data
}



\item
\DIFadd{Sets baselineStats ← }{\DIFadd{mean: μ, std: σ}}



\item
\DIFadd{If data contains invalid values (e.g., NaN), raises NaNError, and baselineStats remains unchanged
}



\end{itemize}




\end{itemize}






\DIFaddend \subsubsection{Access Routine Semantics}
\begin{itemize}
  \item validate input(raw data: $\mathbb{D}$):
    \subitem Validation Steps:
        \subsubitem Check if raw data[data] is of type $\mathbb{R}$[] and non-empty.
        \subsubitem Ensure no NaN or infinite values in the signal.
    \subitem Output:
        \subsubitem Returns True if validation passes.
    \subitem Exceptions:
        \subsubitem InvalidDataError: Non-numeric values or mismatched channel counts.
    \item compute statistics($\mathbb{R}$[]):
    \subitem Output:
        \subsubitem mean, std, min, max and other statistical information
    \subitem Exceptions:
        \subsubitem NaNError if signal contains invalid values after preprocessing
\end{itemize}


\section{MIS of Visualization Module}

\subsection{Module}

Interactive Neuro-imaging Module
\subsection{Uses}

M1 (writePlot), M3 (processed data), M5 (TRF results)


\subsection{Syntax}


\subsubsection{Exported Access Programs}

\begin{center}
\begin{tabular}{p{2cm} p{4cm} p{4cm} p{2cm}}
\hline
\textbf{Name} & \textbf{In} & \textbf{Out} & \textbf{Exceptions} \\
\hline
plot & \DIFdelbegin \DIFdel{List}\DIFdelend \DIFaddbegin \DIFadd{Dict}\DIFaddend <\DIFdelbegin \DIFdel{$\mathbb{R}$}%DIFDELCMD < []%%%
\DIFdel{> (dipole currents),List<}\DIFdelend \DIFaddbegin \DIFadd{$\mathbb{S}$,}\DIFaddend $\mathbb{S}$> (plot type\DIFaddbegin \DIFadd{, plot data path}\DIFaddend ), $\mathbb{S}$(interaction type) & void & Plotting Error \\

\hline
\end{tabular}
\end{center}

\subsection{Semantics}

\subsubsection{State Variables}




\DIFaddbegin \subsection*{\textbf{\DIFadd{activePlots}}}





\DIFaddend \begin{itemize}
\item
\DIFdelbegin \DIFdel{activePlots:
$\mathbb{D}$<PlotID, PlotData> (metadata for open plots)
.
}\DIFdelend \DIFaddbegin \textbf{\DIFadd{Initial State}}\DIFadd{:
An empty dictionary
}\item
\textbf{\DIFadd{Updated By}}\DIFadd{:
}\DIFaddend




\DIFaddbegin \begin{itemize}
\DIFaddend \item
\DIFdelbegin \DIFdel{linkedViews: Set<PlotID>(plots synchronized via linkPlots)
.
}\DIFdelend \DIFaddbegin \DIFadd{plot(data, type)
}\DIFaddend



\DIFdelbegin %DIFDELCMD < \end{itemize}
%DIFDELCMD < %%%
\DIFdelend \DIFaddbegin \item
\DIFadd{update\_plot(plotID, newData)
}\DIFaddend



\DIFdelbegin \subsubsection{\DIFdel{Environment Variables}}
%DIFAUXCMD
\addtocounter{subsubsection}{-1}%DIFAUXCMD
\DIFdelend \DIFaddbegin \item
\DIFadd{close\_plot(plotID)
}\DIFaddend



\DIFdelbegin \DIFdel{GPU acceleration (enabled via plotly.graph\_objects and nilearn.plotting)
}\subsubsection{\DIFdel{Assumptions}}
%DIFAUXCMD
\addtocounter{subsubsection}{-1}%DIFAUXCMD
\DIFdelend \DIFaddbegin \end{itemize}
\DIFaddend




\DIFdelbegin \subsubsection{\DIFdel{Access Routine Semantics}}
%DIFAUXCMD
\addtocounter{subsubsection}{-1}%DIFAUXCMD
%DIFDELCMD <

%DIFDELCMD < \begin{itemize}
%DIFDELCMD < %%%
\DIFdelend \item
\DIFdelbegin \DIFdel{renderCorticalMap()}\DIFdelend \DIFaddbegin \textbf{\DIFadd{Transition}}\DIFaddend :




\begin{itemize}
\item
\DIFdelbegin \textit{\DIFdel{Output}}%DIFAUXCMD
\DIFdel{:
Generates brain activation map using plotly.graph\_objects and nilearn.plotting
}\DIFdelend \DIFaddbegin \textbf{\DIFadd{When a new plot is created}} \DIFadd{via plot, a new entry is added:
}\DIFaddend


\DIFaddbegin \DIFadd{activePlots}[\DIFadd{plotID}] \DIFadd{← PlotData
}



\DIFaddend \item
\DIFdelbegin \textit{\DIFdel{Exceptions}}%DIFAUXCMD
\DIFdel{:
Plotting Error
}\DIFdelend \DIFaddbegin \textbf{\DIFadd{When a plot is updated}}\DIFadd{, the entry is modified:
}\DIFaddend


\DIFaddbegin \DIFadd{activePlots}[\DIFadd{plotID}] \DIFadd{← updatedPlotData
}



\item
\textbf{\DIFadd{When a plot is closed}}\DIFadd{, the entry is removed:
}


\DIFadd{del activePlots}[\DIFadd{plotID}]



\DIFaddend \end{itemize}




\end{itemize}
\DIFaddbegin \subsubsection{\DIFadd{Access Routine: Semantics}}
\DIFaddend

\DIFdelbegin \subsubsection{\DIFdel{Local Functions}}
%DIFAUXCMD
\addtocounter{subsubsection}{-1}%DIFAUXCMD
%DIFDELCMD <

%DIFDELCMD < %%%
\DIFdelend \DIFaddbegin \noindent\textbf{\DIFadd{Syntax:}}
\DIFaddend \begin{itemize}
  \item \DIFdelbegin \DIFdel{\_updateLinkedAxes(}\DIFdelend \DIFaddbegin \texttt{\DIFadd{plot(data: Dict<R,R>}}\DIFadd{,  }\texttt{\DIFadd{interactionMode: String}}\DIFaddend )
\DIFdelbegin \DIFdel{: Propagates axis changes to all linked plots.
  }\DIFdelend \DIFaddbegin \end{itemize}
\DIFaddend

\DIFaddbegin \noindent\textbf{\DIFadd{Valid }\texttt{\DIFadd{plotTypes}} \DIFadd{values (case-insensitive):}}
\begin{itemize}
  \item \texttt{\DIFadd{"time\_series"}} \DIFadd{– Renders time vs amplitude plots for each channel.
  }\item \texttt{\DIFadd{"glass\_brain"}} \DIFadd{– Displays brain activity using anatomical projections (e.g., with }\texttt{\DIFadd{nilearn}}\DIFadd{).
  }\item \texttt{\DIFadd{"topomap"}} \DIFadd{– Sensor-space topographic activation maps.
}\DIFaddend \end{itemize}

\DIFaddbegin \noindent\textbf{\DIFadd{Valid }\texttt{\DIFadd{interactionMode}} \DIFadd{values:}}
\begin{itemize}
  \item \texttt{\DIFadd{"zoom"}} \DIFadd{– Enables scroll and drag-based zooming.
  }\item \texttt{\DIFadd{"drag"}} \DIFadd{– Enable click and drag plots
  }\item \texttt{\DIFadd{"click"}} \DIFadd{– Enable click in data plots
}\end{itemize}
\DIFaddend \newpage

\bibliographystyle {plainnat}
\bibliography {../../../refs/References}

\newpage

\section{Appendix} \label{Appendix}

\subsection{ Dependencies}
\begin{itemize}
\item Plotly: Used for interactive HTML5 visualizations (zooming, panning, tooltips).


\item Nilearn: Handles neuroimaging-specific rendering (glass brain plots).


\end{itemize}
 \DIFdelbegin \subsection{\DIFdel{Performance Notes}}
%DIFAUXCMD
\addtocounter{subsection}{-1}%DIFAUXCMD
%DIFDELCMD < \begin{itemize}
\begin{itemize}%DIFAUXCMD
%DIFDELCMD < \item %%%
\item%DIFAUXCMD
\DIFdel{GPU acceleration is optional but recommended for real-time interaction with large datasets (>10⁶ samples).
}%DIFDELCMD <


\end{itemize}%DIFAUXCMD
%DIFDELCMD < \end{itemize}
%DIFDELCMD <

%DIFDELCMD <  %%%
\DIFdelend\end{document}