\documentclass[12pt, titlepage]{article}

\usepackage{amsmath, mathtools}

\usepackage[round]{natbib}
\usepackage{amsfonts}
\usepackage{amssymb}
\usepackage{graphicx}
\usepackage{colortbl}
\usepackage{xr}
\usepackage{hyperref}
\usepackage{longtable}
\usepackage{xfrac}
\usepackage{tabularx}
\usepackage{float}
\usepackage{siunitx}
\usepackage{booktabs}
\usepackage{multirow}
\usepackage[section]{placeins}
\usepackage{caption}
\usepackage{fullpage}

\hypersetup{
bookmarks=true,     % show bookmarks bar?
colorlinks=true,       % false: boxed links; true: colored links
linkcolor=red,          % color of internal links (change box color with linkbordercolor)
citecolor=blue,      % color of links to bibliography
filecolor=magenta,  % color of file links
urlcolor=cyan          % color of external links
}

\usepackage{array}

\externaldocument{../../SRS/SRS}

%% Comments

\usepackage{color}

\newif\ifcomments\commentstrue %displays comments
%\newif\ifcomments\commentsfalse %so that comments do not display

\ifcomments
\newcommand{\authornote}[3]{\textcolor{#1}{[#3 ---#2]}}
\newcommand{\todo}[1]{\textcolor{red}{[TODO: #1]}}
\else
\newcommand{\authornote}[3]{}
\newcommand{\todo}[1]{}
\fi

\newcommand{\wss}[1]{\authornote{blue}{SS}{#1}} 
\newcommand{\plt}[1]{\authornote{magenta}{TPLT}{#1}} %For explanation of the template
\newcommand{\an}[1]{\authornote{cyan}{Author}{#1}}

%% Common Parts

\newcommand{\progname}{Live Neuro} % PUT YOUR PROGRAM NAME HERE
\newcommand{\authname}{Bo Liang} % AUTHOR NAMES

\usepackage{hyperref}
    \hypersetup{colorlinks=true, linkcolor=blue, citecolor=blue, filecolor=blue,
                urlcolor=blue, unicode=false}
    \urlstyle{same}
                                


\begin{document}

\title{Module Interface Specification for \progname{}}

\author{\authname}

\date{\today}

\maketitle

\pagenumbering{roman}

\section{Revision History}

\begin{tabularx}{\textwidth}{p{3cm}p{2cm}X}
\toprule {\bf Date} & {\bf Version} & {\bf Notes}\\
\midrule
March 17 & 1.0 & Initial Draft\\
\bottomrule
\end{tabularx}

~\newpage

\section{Symbols, Abbreviations and Acronyms}

See SRS Documentation at \href{https://github.com/liang-bo96/CAS741/blob/main/docs/SRS/SRS.pdf}{SRS}


\newpage

\tableofcontents

\newpage

\pagenumbering{arabic}

\section{Introduction}

This document specifies the interfaces for modules in the Live Neuro system, an interactive neural data visualization tool. It complements the
\href{https://github.com/liang-bo96/CAS741/blob/main/docs/SRS/SRS.pdf}{SRS}
and \href{https://github.com/liang-bo96/CAS741/blob/main/docs/Design/SoftArchitecture/MG.pdf}{MG}, with full implementation details available at
\href{https://github.com/liang-bo96/CAS741}{ GitHub Repository}.

\section{Notation}


The following table summarizes the primitive data types used by \progname.

\begin{center}
\renewcommand{\arraystretch}{1.2}
\noindent
\begin{tabular}{l l p{7.5cm}}
\toprule
\textbf{Data Type} & \textbf{Notation} & \textbf{Description}\\
\midrule
String & $\mathbb{S}$ & a character string \\
data type dictionary & $\mathbb{D}$ & a storage type for KV structures \\
bool & $\mathbb{B}$ &  Boolean data type, which can hold one of two values: either true or false \\
integer & $\mathbb{Z}$ & a number without a fractional component in (-$\infty$, $\infty$) \\
natural number & $\mathbb{N}$ & a number without a fractional component in [1, $\infty$) \\
real & $\mathbb{R}$ & any number in (-$\infty$, $\infty$)\\
\bottomrule
\end{tabular}
\end{center}

\noindent
The specification of \progname \ uses some derived data types: sequences, strings, and
tuples. Sequences are lists filled with elements of the same data type. Strings
are sequences of characters. Tuples contain a list of values, potentially of
different types. In addition, \progname \ uses functions, which
are defined by the data types of their inputs and outputs. Local functions are
described by giving their type signature followed by their specification.

\section{Module Decomposition}

The following table is taken directly from the Module Guide document for this project.

\begin{table}[h!]
\centering
\begin{tabular}{p{0.3\textwidth} p{0.5\textwidth}p{0.1\textwidth}}
\toprule
\textbf{Level 1} & \textbf{Level 2}& \textbf{Module ID}\\
\midrule

{Hardware-Hiding Module} &  Hardware-Hiding Module
 & M1 \\
\midrule

\multirow{3}{0.3\textwidth}{Behaviour-Hiding Module}
& Input Format Module & M2\\
& Data Processing Module & M3\\
& Visualization Module & M4\\

\midrule

\multirow{1}{0.3\textwidth}{Software Decision Module} & TRF Calculation Module & M5\\
\bottomrule

\end{tabular}
\caption{Module Hierarchy}
\label{TblMH}
\end{table}


\newpage
~\newpage

\section{MIS of Hardware-Hiding Module(M1)}

\subsection{Module}
M1: OS Abstraction Layer

\subsection{Uses}

\item Directly interacts with OS APIs (e.g., file I/O, hardware drivers).


\subsection{Syntax}

\subsubsection{Exported Access Programs}

\begin{center}
\begin{tabular}{p{2cm} p{4cm} p{4cm} p{4cm}}
\hline
\textbf{Name} & \textbf{In} & \textbf{Out} & \textbf{Exceptions} \\
\hline
readFile & $\mathbb{S}$ (file path) & Sequence & FileNotFoundError \\
savePlot & plot data & file(.jpeg) & SavePlotError \\
\hline
\end{tabular}
\end{center}

\subsection{Semantics}

\subsubsection{Environment Variables}

File system, display hardware.
\subsubsection{Assumptions}
OS compatibility (Linux/Windows/macOS)

\subsubsection{Access Routine Semantics}

\begin{itemize}
\item readFile(): Reads neural data from disk, returns a sequence of $\mathbb{R}$ numbers.


\item savePlot(): Renders visualization output to screen or file.


\end{itemize}




\section{MIS of Input Format Module}

\subsection{Module}
Multi-Format MEG/EEG Data Parser

\subsection{Uses}

\item M1 (readFile for raw data loading).

\subsection{Syntax}

\subsubsection{Exported Constants}

supported formats = [EDF, FIF, BrainVision](supported data formats).\\
max channels = 256(maximum allowed channels per dataset).


\subsubsection{Exported Access Programs}

\begin{center}
\begin{tabular}{p{2cm} p{4cm} p{4cm} p{2cm}}
\hline
\textbf{Name} & \textbf{In} & \textbf{Out} & \textbf{Exceptions} \\
\hline
load edf() & $\mathbb{S}$ (EDF file path) & $\mathbb{D}$<$\mathbb{S}$: $\mathbb{R}$[]> & EDFFileError \\
load fif() & $\mathbb{S}$ (FIF file path) & $\mathbb{D}$<$\mathbb{S}$: $\mathbb{R}$[]> & FIFFileError \\
load brainvision() & $\mathbb{S}$ (.vhdr file path) & $\mathbb{D}$<$\mathbb{S}$: $\mathbb{R}$[]> & VHDRFileError \\
\hline
\end{tabular}
\end{center}

\subsection{Semantics}

\subsubsection{State Variables}

\begin{itemize}
\item currentFormat: $\mathbb{S}$(last detected data format, e.g., "FIF").


\item metadataCache: $\mathbb{D}$(cached metadata from parsed files).


\end{itemize}


\subsubsection{Environment Variables}


\subsubsection{Access Routine Semantics}
\begin{itemize}
  \item load edf():
    \subitem Output:
        \subsubitem data: Time-series array of shape[channels × samples].
    \subitem Exception:
        \subsubitem EDFHeaderError: Invalid EDF header structure.
    \subitem Implementation: Uses eelbrain for EDF parsing.
  \item load fif():
    \subitem Output:
        \subsubitem data: MEG/EEG sensor data.
    \subitem Exception:
        \subsubitem FIFFileError: FIF file version mismatch or missing data tags.
    \subitem Implementation: Relies on mne-python library.
  \item load brainvision():
    \subitem Output:
        \subsubitem data: EEG data segmented by markers
    \subitem Exception:
        \subsubitem VHDRParseError: Inconsistent header fields in.vhdr file.
\end{itemize}


\subsubsection{Local Functions}
\begin{itemize}
\item parse edf header(): Extracts EDF header fields and validates integrity.\\
\item align brainvision files(): Synchronizes.vhdr,.vmrk, and.eeg data.
\end{itemize}

\section{MIS of Data Processing Module}

\subsection{Module}

Statistical Preprocessing and Validation\\
Primary Function: Validates input data integrity and applies statistical preprocessing to neural signals.
\subsection{Uses}

\begin{itemize}
\item M2(Input Format Module): Receives parsed raw data (e.g., MEG/EEG time-series).


\item M5(TRF Calculation Module): Provides preprocessed data for dipole current computation.


\end{itemize}



\subsection{Syntax}


\subsubsection{Exported Access Programs}

\begin{center}
\begin{tabular}{p{2cm} p{4cm} p{4cm} p{2cm}}
\hline
\textbf{Name} & \textbf{In} & \textbf{Out} & \textbf{Exceptions} \\
\hline
validate input & $\mathbb{D}$<$\mathbb{S}$: $\mathbb{R}$[]> & $\mathbb{B}$ & InvalidDataError \\
compute statistics & $\mathbb{R}$[] & $\mathbb{D}$<$\mathbb{S}$: $\mathbb{R}$> & NaNError \\

\hline
\end{tabular}
\end{center}

\subsection{Semantics}

\subsubsection{State Variables}
\begin{itemize}
\item validatedSignals: $\mathbb{D}$<$\mathbb{S}$: $\mathbb{R}$[]>(cached validated data).


\item baselineStats: $\mathbb{D}$<$\mathbb{S}$: $\mathbb{R}$>(mean and std of reference signals).


\end{itemize}

\subsubsection{Access Routine Semantics}
\begin{itemize}
  \item validate input(raw data: $\mathbb{D}$):
    \subitem Validation Steps:
        \subsubitem Check if raw data[data] is of type $\mathbb{R}$[] and non-empty.
        \subsubitem Ensure no NaN or infinite values in the signal.
    \subitem Output:
        \subsubitem Returns True if validation passes.
    \subitem Exceptions:
        \subsubitem InvalidDataError: Non-numeric values or mismatched channel counts.
    \item compute statistics($\mathbb{R}$[]):
    \subitem Output:
        \subsubitem mean, std, min, max and other statistical information
    \subitem Exceptions:
        \subsubitem NaNError if signal contains invalid values after preprocessing
\end{itemize}


\section{MIS of Visualization Module}

\subsection{Module}

Interactive Neuro-imaging Module
\subsection{Uses}

M1 (writePlot), M3 (processed data), M5 (TRF results)


\subsection{Syntax}


\subsubsection{Exported Access Programs}

\begin{center}
\begin{tabular}{p{2cm} p{4cm} p{4cm} p{2cm}}
\hline
\textbf{Name} & \textbf{In} & \textbf{Out} & \textbf{Exceptions} \\
\hline
plot & List<$\mathbb{R}$[]> (dipole currents), List<$\mathbb{S}$>(plot type), $\mathbb{S}$(interaction type) & void & Plotting Error \\

\hline
\end{tabular}
\end{center}

\subsection{Semantics}

\subsubsection{State Variables}

\begin{itemize}
\item activePlots: $\mathbb{D}$<PlotID, PlotData> (metadata for open plots).


\item linkedViews: Set<PlotID>(plots synchronized via linkPlots).


\end{itemize}


\subsubsection{Environment Variables}

GPU acceleration (enabled via plotly.graph\_objects and nilearn.plotting)



\subsubsection{Access Routine: Semantics}

\noindent\textbf{Syntax:}
\begin{itemize}
  \item \texttt{plot(data: List<R[]>}, \texttt{plotTypes: List<String>}, \texttt{interactionMode: String})
\end{itemize}

\noindent\textbf{Valid \texttt{plotTypes} values (case-insensitive):}
\begin{itemize}
  \item \texttt{"time\_series"} – Renders time vs amplitude plots for each channel.
  \item \texttt{"glass\_brain"} – Displays brain activity using anatomical projections (e.g., with \texttt{nilearn}).
  \item \texttt{"topomap"} – Sensor-space topographic activation maps.
\end{itemize}

\noindent\textbf{Valid \texttt{interactionMode} values:}
\begin{itemize}
  \item \texttt{"zoom"} – Enables scroll and drag-based zooming.
  \item \texttt{"drag"} – Enable click and drag plots
  \item \texttt{"click"} – Enable click in data plots
\end{itemize}





\subsubsection{Local Functions}

\begin{itemize}
\item \_updateLinkedAxes(): Propagates axis changes to all linked plots.


\end{itemize}

\newpage

\bibliographystyle {plainnat}
\bibliography {../../../refs/References}

\newpage

\section{Appendix} \label{Appendix}

\subsection{ Dependencies}
\begin{itemize}
\item Plotly: Used for interactive HTML5 visualizations (zooming, panning, tooltips).


\item Nilearn: Handles neuroimaging-specific rendering (glass brain plots).


\end{itemize}
\subsection{Performance Notes}
\begin{itemize}
\item GPU acceleration is optional but recommended for real-time interaction with large datasets (>10⁶ samples).


\end{itemize}

\end{document}