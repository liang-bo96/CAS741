\documentclass[12pt, titlepage]{article}

\usepackage{amsmath, mathtools}

\usepackage[round]{natbib}
\usepackage{amsfonts}
\usepackage{amssymb}
\usepackage{graphicx}
\usepackage{colortbl}
\usepackage{xr}
\usepackage{hyperref}
\usepackage{longtable}
\usepackage{xfrac}
\usepackage{tabularx}
\usepackage{float}
\usepackage{siunitx}
\usepackage{booktabs}
\usepackage{multirow}
\usepackage[section]{placeins}
\usepackage{caption}
\usepackage{fullpage}

\hypersetup{
bookmarks=true,     % show bookmarks bar?
colorlinks=true,       % false: boxed links; true: colored links
linkcolor=red,          % color of internal links (change box color with linkbordercolor)
citecolor=blue,      % color of links to bibliography
filecolor=magenta,  % color of file links
urlcolor=cyan          % color of external links
}

\usepackage{array}

\externaldocument{../../SRS/SRS}

\input{../../Comments}
%% Common Parts

\newcommand{\progname}{Live Neuro} % PUT YOUR PROGRAM NAME HERE
\newcommand{\authname}{Bo Liang} % AUTHOR NAMES

\usepackage{hyperref}
    \hypersetup{colorlinks=true, linkcolor=blue, citecolor=blue, filecolor=blue,
                urlcolor=blue, unicode=false}
    \urlstyle{same}
                                


\begin{document}

\title{Module Interface Specification for \progname{}}

\author{\authname}

\date{\today}

\maketitle

\pagenumbering{roman}

\section{Revision History}

\begin{tabularx}{\textwidth}{p{3cm}p{2cm}X}
\toprule {\bf Date} & {\bf Version} & {\bf Notes}\\
\midrule
March 17 & 1.0 & Initial Draft\\
Apr 11 & 1.1 & Modify\\
\bottomrule
\end{tabularx}

~\newpage

\section{Symbols, Abbreviations and Acronyms}

See SRS Documentation at \href{https://github.com/liang-bo96/CAS741/blob/main/docs/SRS/SRS.pdf}{SRS}


\newpage

\tableofcontents

\newpage

\pagenumbering{arabic}

\section{Introduction}

This document specifies the interfaces for modules in the Live Neuro system, an interactive neural data visualization tool. It complements the
\href{https://github.com/liang-bo96/CAS741/blob/main/docs/SRS/SRS.pdf}{SRS}
and \href{https://github.com/liang-bo96/CAS741/blob/main/docs/Design/SoftArchitecture/MG.pdf}{MG}, with full implementation details available at
\href{https://github.com/liang-bo96/CAS741}{ GitHub Repository}.

\section{Notation}


The following table summarizes the primitive data types used by \progname.

\begin{center}
\renewcommand{\arraystretch}{1.2}
\noindent
\begin{tabular}{l l p{7.5cm}}
\toprule
\textbf{Data Type} & \textbf{Notation} & \textbf{Description}\\
\midrule
String & $\mathbb{S}$ & a character string \\
Dict[str, Any]
 & $\mathbb{D}$ & a storage type for KV structures \\
bool & $\mathbb{B}$ &  Boolean data type, which can hold one of two values: either true or false \\
integer & $\mathbb{Z}$ & a number without a fractional component in (-$\infty$, $\infty$) \\
natural number & $\mathbb{N}$ & a number without a fractional component in [1, $\infty$) \\
real & $\mathbb{R}$ & any number in (-$\infty$, $\infty$)\\
\bottomrule
\end{tabular}
\end{center}

\noindent
The specification of \progname \ uses some derived data types: sequences, strings, and
tuples. Sequences are lists filled with elements of the same data type. Strings
are sequences of characters. Tuples contain a list of values, potentially of
different types. In addition, \progname \ uses functions, which
are defined by the data types of their inputs and outputs. Local functions are
described by giving their type signature followed by their specification.

\section{Module Decomposition}

The following table is taken directly from the Module Guide document for this project.

\begin{table}[h!]
\centering
\begin{tabular}{p{0.3\textwidth} p{0.5\textwidth}p{0.1\textwidth}}
\toprule
\textbf{Level 1} & \textbf{Level 2}& \textbf{Module ID}\\
\midrule

{Hardware-Hiding Module} &  Hardware-Hiding Module
 & M1 \\
\midrule

\multirow{3}{0.3\textwidth}{Behaviour-Hiding Module}
& Input Format Module & M2\\
& Data Processing Module & M3\\
& Visualization Module & M4\\

\midrule

\multirow{1}{0.3\textwidth}{Software Decision Module} & TRF Calculation Module & M5\\
\bottomrule

\end{tabular}
\caption{Module Hierarchy}
\label{TblMH}
\end{table}


\newpage
~\newpage

\section{MIS of Hardware-Hiding Module(M1)}

\subsection{Module}
M1: OS Abstraction Layer

\subsection{Uses}

\item Directly interacts with OS APIs (e.g., file I/O, hardware drivers).


\subsection{Syntax}

\subsubsection{Exported Access Programs}

\begin{center}
\begin{tabular}{p{2cm} p{4cm} p{4cm} p{4cm}}
\hline
\textbf{Name} & \textbf{In} & \textbf{Out} & \textbf{Exceptions} \\
\hline
readFile & $\mathbb{S}$ (file path) & Sequence & FileNotFoundError \\
savePlot & plot data & file(.jpeg) & SavePlotError \\
\hline
\end{tabular}
\end{center}

\subsection{Semantics}

\subsubsection{Environment Variables}

File system-fs, display hardware-dh.
\subsubsection{Assumptions}
OS compatibility (Linux/Windows/macOS)

\subsubsection{Access Routine Semantics}

\begin{itemize}
\item readFile(): Reads neural data from disk, returns a sequence of characters.


\item savePlot(): Renders visualization output to screen or file under ../visualization/output.


\end{itemize}

\newpage


\section{MIS of Input Format Module}

\subsection{Module}
Multi-Format MEG/EEG Data Parser

\subsection{Uses}

\item M1 (readFile for raw data loading).

\subsection{Syntax}

\subsubsection{Exported Constants}

supported formats = [EDF, FIF] (supported data formats).\\
max channels = 256 (maximum allowed channels per dataset).


\subsubsection{Exported Access Programs}

\begin{center}
\begin{tabular}{p{2cm} p{4cm} p{4cm} p{2cm}}
\hline
\textbf{Name} & \textbf{In} & \textbf{Out} & \textbf{Exceptions} \\
\hline
load edf() & $\mathbb{S}$ (EDF file path) & $\mathbb{D}$<$\mathbb{S}$: $\mathbb{R}$[]> & EDFFileError \\
load fif() & $\mathbb{S}$ (FIF file path) & $\mathbb{D}$<$\mathbb{S}$: $\mathbb{R}$[]> & FIFFileError \\
load brainvision() & $\mathbb{S}$ (.vhdr file path) & $\mathbb{D}$<$\mathbb{S}$: $\mathbb{R}$[]> & VHDRFileError \\
\hline
\end{tabular}
\end{center}

\textbf{Notation Explanation:}


\begin{itemize}


 \item \texttt{$\mathbb{R}$[]} – a one-dimensional array of real numbers (e.g., one EEG channel)


\end{itemize}



\subsection{Semantics}

\subsubsection{State Variables}

\begin{itemize}
\item currentFormat: $\mathbb{S}$(last detected data format, e.g., FIF(enum type)).



\begin{itemize}
\item
Stores the most recently detected data format.


FormatType is an \textbf{enumerated type} with the following values:




\begin{itemize}
\item
FIF



\item
EDF



\end{itemize}




\end{itemize}





\end{itemize}


\subsubsection{Environment Variables}




\begin{itemize}
\item
\textbf{DATA PATH: String}\textbf{}


Path to the directory where data files are stored.


This value can be passed explicitly or set using an environment variable (e.g., LIVE NEURO DATA PATH).



\end{itemize}




\subsubsection{Access Routine Semantics}
\begin{itemize}




\subsubsection*{\textbf{load\_edf(filepath: String)}





\begin{itemize}
\item
\textbf{Purpose}: Loads data from an EDF (European Data Format) file and returns EEG time-series data.



\item
\textbf{Input}:




\begin{itemize}
\item
filepath: Path to the .edf file as a string.



\end{itemize}




\item
\textbf{Output}:




\begin{itemize}
\item
data: A data Dict[String : array]



\end{itemize}




\item
\textbf{Exceptions}:




\begin{itemize}
\item
EDFFileError: Raised if the EDF file is malformed or unsupported.



\end{itemize}




\item
\textbf{Implementation Detail}: Uses mne.io.read\_raw\_edf()

\end{itemize}









\subsubsection*{\textbf{load\_fif(filepath: String) → mne.io.Raw}}





\begin{itemize}
\item
\textbf{Purpose}: Loads MEG/EEG sensor data from an MNE .fif file.



\item
\textbf{Input}:




\begin{itemize}
\item
filepath: Path to the .fif file as a string.



\end{itemize}




\item
\textbf{Output}:




\begin{itemize}
\item
data: A Raw object from the mne library.



\end{itemize}




\item
\textbf{Exceptions}:




\begin{itemize}
\item
FIFFileError: Raised for version mismatch, file corruption, or missing tags.



\end{itemize}




\item
\textbf{Implementation Detail}: Calls mne.io.read\_raw\_fif().



\end{itemize}




\end{itemize}




\section{MIS of Data Processing Module}

\subsection{Module}

Statistical Preprocessing and Validation\\
Primary Function: Validates input data integrity and applies statistical preprocessing to neural signals.
\subsection{Uses}

\begin{itemize}
\item M2(Input Format Module): Receives parsed raw data (e.g., MEG/EEG time-series).


\item M5(TRF Calculation Module): Provides preprocessed data for dipole current computation.


\end{itemize}



\subsection{Syntax}


\subsubsection{Exported Access Programs}

\begin{center}
\begin{tabular}{p{2cm} p{4cm} p{4cm} p{2cm}}
\hline
\textbf{Name} & \textbf{In} & \textbf{Out} & \textbf{Exceptions} \\
\hline
validate input & $\mathbb{D}$<$\mathbb{S}$: $\mathbb{R}$[]> & $\mathbb{B}$ & InvalidDataError \\
compute statistics & $\mathbb{R}$[] & $\mathbb{D}$<$\mathbb{S}$: $\mathbb{R}$> & NaNError \\

\hline
\end{tabular}
\end{center}

Although input validation could technically be performed immediately after data is loaded, we intentionally separate the \textbf{validation logic} into its own module for reasons of \textbf{modularity}, \textbf{reusability}, and \textbf{maintainability}.


Specifically:


\begin{itemize}
\item
\textbf{Validation is not only for file loading}: In many use cases, data may come from \textbf{non-file sources}, such as the \textbf{internet}, \textbf{APIs}, or \textbf{in-memory structures}. Centralizing validation ensures consistency across all entry points.

\item
\textbf{Validation rules evolve}: Scientific or clinical datasets may have \textbf{changing standards} or \textbf{conditional rules}. Keeping validation separate allows updates to these rules \textbf{without modifying the loader} or tightly coupled modules.

\item
\textbf{Cleaner architecture}: Following the \textbf{Single Responsibility Principle}, the file-loading module focuses purely on parsing and decoding data formats (e.g., EDF, FIF), while the validation module handles semantic correctness and integrity.

\item
\textbf{Testing and reuse}: A standalone validation module can be tested independently and reused in \textbf{preprocessing pipelines}, \textbf{live data streaming}, or even \textbf{user-uploaded data checks}.

\end{itemize}





Therefore, validation is performed in a \textbf{later module}, after the raw data is loaded, to support broader use cases and maintain long-term flexibility of the system.

\subsection{Semantics}

\subsubsection{State Variables}



\subsubsection*{\textbf{validatedSignals}}





\begin{itemize}
\item
\textbf{Initial State}: None



\item
\textbf{Updated By}: validate\_input(data)



\item
\textbf{Transition}:




\begin{itemize}
\item
If data passes validation, validatedSignals ← data



\item
If validation fails, validatedSignals remains unchanged



\end{itemize}




\end{itemize}








\subsubsection*{\textbf{baselineStats}}





\begin{itemize}
\item
\textbf{Initial State}: None



\item
\textbf{Updated By}: compute\_statistics(data)



\item
\textbf{Transition}:




\begin{itemize}
\item
Computes mean and standard deviation from data



\item
Sets baselineStats ← {mean: μ, std: σ}



\item
If data contains invalid values (e.g., NaN), raises NaNError, and baselineStats remains unchanged



\end{itemize}




\end{itemize}






\subsubsection{Access Routine Semantics}
\begin{itemize}
  \item validate input(raw data: $\mathbb{D}$):
    \subitem Validation Steps:
        \subsubitem Check if raw data[data] is of type $\mathbb{R}$[] and non-empty.
        \subsubitem Ensure no NaN or infinite values in the signal.
    \subitem Output:
        \subsubitem Returns True if validation passes.
    \subitem Exceptions:
        \subsubitem InvalidDataError: Non-numeric values or mismatched channel counts.
    \item compute statistics($\mathbb{R}$[]):
    \subitem Output:
        \subsubitem mean, std, min, max and other statistical information
    \subitem Exceptions:
        \subsubitem NaNError if signal contains invalid values after preprocessing
\end{itemize}


\section{MIS of Visualization Module}

\subsection{Module}

Interactive Neuro-imaging Module
\subsection{Uses}

M1 (writePlot), M3 (processed data), M5 (TRF results)


\subsection{Syntax}


\subsubsection{Exported Access Programs}

\begin{center}
\begin{tabular}{p{2cm} p{4cm} p{4cm} p{2cm}}
\hline
\textbf{Name} & \textbf{In} & \textbf{Out} & \textbf{Exceptions} \\
\hline
plot & Dict<$\mathbb{S}$,$\mathbb{S}$> (plot type, plot data path), $\mathbb{S}$(interaction type) & void & Plotting Error \\

\hline
\end{tabular}
\end{center}

\subsection{Semantics}

\subsubsection{State Variables}




\subsection*{\textbf{activePlots}}





\begin{itemize}
\item
\textbf{Initial State}:
An empty dictionary
\item
\textbf{Updated By}:




\begin{itemize}
\item
plot(data, type)



\item
update\_plot(plotID, newData)



\item
close\_plot(plotID)



\end{itemize}




\item
\textbf{Transition}:




\begin{itemize}
\item
\textbf{When a new plot is created} via plot, a new entry is added:


activePlots[plotID] ← PlotData



\item
\textbf{When a plot is updated}, the entry is modified:


activePlots[plotID] ← updatedPlotData



\item
\textbf{When a plot is closed}, the entry is removed:


del activePlots[plotID]



\end{itemize}




\end{itemize}
\subsubsection{Access Routine: Semantics}

\noindent\textbf{Syntax:}
\begin{itemize}
  \item \texttt{plot(data: Dict<R,R>},  \texttt{interactionMode: String})
\end{itemize}

\noindent\textbf{Valid \texttt{plotTypes} values (case-insensitive):}
\begin{itemize}
  \item \texttt{"time\_series"} – Renders time vs amplitude plots for each channel.
  \item \texttt{"glass\_brain"} – Displays brain activity using anatomical projections (e.g., with \texttt{nilearn}).
  \item \texttt{"topomap"} – Sensor-space topographic activation maps.
\end{itemize}

\noindent\textbf{Valid \texttt{interactionMode} values:}
\begin{itemize}
  \item \texttt{"zoom"} – Enables scroll and drag-based zooming.
  \item \texttt{"drag"} – Enable click and drag plots
  \item \texttt{"click"} – Enable click in data plots
\end{itemize}
\newpage

\bibliographystyle {plainnat}
\bibliography {../../../refs/References}

\newpage

\section{Appendix} \label{Appendix}

\subsection{ Dependencies}
\begin{itemize}
\item Plotly: Used for interactive HTML5 visualizations (zooming, panning, tooltips).


\item Nilearn: Handles neuroimaging-specific rendering (glass brain plots).


\end{itemize}
\end{document}