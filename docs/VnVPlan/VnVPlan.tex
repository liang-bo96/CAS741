% Options for packages loaded elsewhere
\PassOptionsToPackage{unicode}{hyperref}
\PassOptionsToPackage{hyphens}{url}
%
\documentclass[
]{article}
\usepackage{amsmath,amssymb}
\usepackage{lmodern}
\usepackage{iftex}
\ifPDFTeX
  \usepackage[T1]{fontenc}
  \usepackage[utf8]{inputenc}
  \usepackage{textcomp} % provide euro and other symbols
\else % if luatex or xetex
  \usepackage{unicode-math}
  \defaultfontfeatures{Scale=MatchLowercase}
  \defaultfontfeatures[\rmfamily]{Ligatures=TeX,Scale=1}
\fi
% Use upquote if available, for straight quotes in verbatim environments
\IfFileExists{upquote.sty}{\usepackage{upquote}}{}
\IfFileExists{microtype.sty}{% use microtype if available
  \usepackage[]{microtype}
  \UseMicrotypeSet[protrusion]{basicmath} % disable protrusion for tt fonts
}{}
\makeatletter
\@ifundefined{KOMAClassName}{% if non-KOMA class
  \IfFileExists{parskip.sty}{%
    \usepackage{parskip}
  }{% else
    \setlength{\parindent}{0pt}
    \setlength{\parskip}{6pt plus 2pt minus 1pt}}
}{% if KOMA class
  \KOMAoptions{parskip=half}}
\makeatother
\usepackage{xcolor}
\usepackage{longtable,booktabs,array}
\usepackage{calc} % for calculating minipage widths
% Correct order of tables after \paragraph or \subparagraph
\usepackage{etoolbox}
\makeatletter
\patchcmd\longtable{\par}{\if@noskipsec\mbox{}\fi\par}{}{}
\makeatother
% Allow footnotes in longtable head/foot
\IfFileExists{footnotehyper.sty}{\usepackage{footnotehyper}}{\usepackage{footnote}}
\makesavenoteenv{longtable}
\setlength{\emergencystretch}{3em} % prevent overfull lines
\providecommand{\tightlist}{%
  \setlength{\itemsep}{0pt}\setlength{\parskip}{0pt}}
\setcounter{secnumdepth}{-\maxdimen} % remove section numbering
\ifLuaTeX
  \usepackage{selnolig}  % disable illegal ligatures
\fi
\IfFileExists{bookmark.sty}{\usepackage{bookmark}}{\usepackage{hyperref}}
\IfFileExists{xurl.sty}{\usepackage{xurl}}{} % add URL line breaks if available
\urlstyle{same} % disable monospaced font for URLs
\hypersetup{
  pdftitle={System Verification and Validation Plan for Live Neuro},
  hidelinks,
  pdfcreator={LaTeX via pandoc}}

\title{System Verification and Validation Plan for Live Neuro}
\author{}
\date{}

\begin{document}
\maketitle

\begin{quote}
Bo Liang February 26, 2025

\textbf{Revision History}

\textbf{Date Version Notes}

February26, 2025 1.0 Initial Draft

\textbf{Contents}
\end{quote}

\hypertarget{table-of-contents}{%
\section{Table of Contents}\label{table-of-contents}}

\protect\hyperlink{symbols-abbreviations-and-acronyms}{1. Symbols,
Abbreviations, and Acronyms
\protect\hyperlink{symbols-abbreviations-and-acronyms}{3}}

\protect\hyperlink{general-information}{2. General Information
\protect\hyperlink{general-information}{5}}

\protect\hyperlink{summary}{2.1 Summary \protect\hyperlink{summary}{5}}

\protect\hyperlink{objectives}{2.2 Objectives
\protect\hyperlink{objectives}{5}}

\protect\hyperlink{challenge-level-and-extras}{2.3 Challenge Level and
Extras \protect\hyperlink{challenge-level-and-extras}{5}}

\protect\hyperlink{relevant-documentation}{2.4 Relevant Documentation
\protect\hyperlink{relevant-documentation}{5}}

\protect\hyperlink{plan}{3. Plan \protect\hyperlink{plan}{6}}

\protect\hyperlink{verification-and-validation-team}{3.1 Verification
and Validation Team
\protect\hyperlink{verification-and-validation-team}{6}}

\protect\hyperlink{srs-verification-plan}{3.2 SRS Verification Plan
\protect\hyperlink{srs-verification-plan}{6}}

\protect\hyperlink{design-verification-plan}{3.3 Design Verification
Plan \protect\hyperlink{design-verification-plan}{6}}

\protect\hyperlink{_Toc191500205}{3.4 Implementation Verification Plan
\protect\hyperlink{_Toc191500205}{7}}

\protect\hyperlink{automated-testing-and-verification-tools}{3.5
Automated Testing and Verification Tools
\protect\hyperlink{automated-testing-and-verification-tools}{7}}

\protect\hyperlink{software-validation-plan}{3.6 Software Validation
Plan \protect\hyperlink{software-validation-plan}{7}}

\protect\hyperlink{system-tests}{4. System Tests
\protect\hyperlink{system-tests}{8}}

\protect\hyperlink{tests-for-functional-requirements}{4.1 Tests for
Functional Requirements
\protect\hyperlink{tests-for-functional-requirements}{8}}

\protect\hyperlink{tests-for-nonfunctional-requirements}{4.2 Tests for
Nonfunctional Requirements
\protect\hyperlink{tests-for-nonfunctional-requirements}{8}}

\protect\hyperlink{traceability-between-test-cases-and-requirements}{4.3
Traceability Between Test Cases and Requirements
\protect\hyperlink{traceability-between-test-cases-and-requirements}{11}}

\protect\hyperlink{unit-test-description}{5. Unit Test Description
\protect\hyperlink{unit-test-description}{12}}

\protect\hyperlink{unit-testing-scope}{5.1 Unit Testing Scope
\protect\hyperlink{unit-testing-scope}{12}}

\protect\hyperlink{tests-for-functional-requirements-1}{5.2 Tests for
Functional Requirements
\protect\hyperlink{tests-for-functional-requirements-1}{12}}

\protect\hyperlink{tests-for-nonfunctional-requirements-1}{5.3 Tests for
Nonfunctional Requirements
\protect\hyperlink{tests-for-nonfunctional-requirements-1}{13}}

\protect\hyperlink{traceability-between-test-cases-and-modules}{5.4
Traceability Between Test Cases and Modules
\protect\hyperlink{traceability-between-test-cases-and-modules}{14}}

\protect\hyperlink{usability-survey-questionnaire}{6. Usability Survey
Questionnaire \protect\hyperlink{usability-survey-questionnaire}{16}}

\hypertarget{symbols-abbreviations-and-acronyms}{%
\section{Symbols, Abbreviations, and
Acronyms}\label{symbols-abbreviations-and-acronyms}}

The definition of symbols, abbreviations and acronyms is as same as
those in my SRS documents.

This document presents the Verification and Validation (VnV) Plan for
Live Neuro, an interactive neural data visualization tool. The plan is
structured into four main sections:

1. General Information

• Provides an overview of Live Neuro, its objectives, challenge level,
and relevant documentation.

• Defines the scope of verification, including functional and
non-functional requirements.

2. Plan

• Describes the verification strategies for different project phases:
SRS, Design, VnV, and Implementation.

• Details the roles and responsibilities of the Verification and
Validation Team.

• Includes plans for automated testing, validation tools, and software
validation.

3. System Tests

• Verifies both functional and non-functional requirements at the system
level.

• Defines test cases for acoustic stimulus processing, interactive
visualization, and cross-platform compatibility.

• Ensures the system behaves as expected through automated and manual
testing.

4. Unit Test Description

• Focuses on validating individual modules such as data preprocessing,
TRF model computation, and chart rendering.

• Covers both functional and non-functional aspects with unit test
cases.

• Includes traceability analysis to ensure proper coverage of
requirements.

This VnV plan provides a structured and comprehensive approach to
verifying and validating Live Neuro, ensuring reliability, accuracy, and
usability before deployment.

\begin{enumerate}
\def\labelenumi{\arabic{enumi}.}
\setcounter{enumi}{1}
\item ~
  \hypertarget{general-information}{%
  \section{General Information}\label{general-information}}

  \begin{enumerate}
  \def\labelenumii{\arabic{enumii}.}
  \item ~
    \hypertarget{summary}{%
    \subsection{Summary}\label{summary}}
  \end{enumerate}
\end{enumerate}

Live Neuro is an interactive neural data visualization tool designed to
display EEG and MEG data through linked multidimensional charts. It
supports continuous response analysis for acoustic and semantic stimuli.
This plan aims to verify its functional correctness, interactivity,
cross-platform compatibility, and user-friendliness.

\hypertarget{objectives}{%
\subsection{Objectives}\label{objectives}}

\textbf{Primary Objectives:}

• Verify the implementation of functional requirements and
non-functional requirements.

• Ensure that data visualization results meet scientific standards
(NFR1).

• Validate cross-platform compatibility (Linux, Windows 10+, macOS).

\textbf{Non-Objectives:}

• Do not validate the internal implementation of third-party libraries
(e.g., Matplotlib).

• Do not conduct large-scale performance stress testing due to resource
limitations.

\hypertarget{challenge-level-and-extras}{%
\subsection{Challenge Level and
Extras}\label{challenge-level-and-extras}}

\textbf{Challenge Level:}

• Advanced (requires support for complex interactive charts and
multi-platform adaptation).

\textbf{Additional Content:}

• User usability testing (feedback from users with a neuroscience
background).

• Static code analysis (using flake8 and pylint).

\hypertarget{relevant-documentation}{%
\subsection{Relevant Documentation}\label{relevant-documentation}}

• \textbf{SRS Document}: Defines requirements and system models.

• \textbf{User Manual}: Operation guide and test configuration
instructions (to be supplemented).

\begin{enumerate}
\def\labelenumi{\arabic{enumi}.}
\setcounter{enumi}{2}
\item ~
  \hypertarget{plan}{%
  \section{Plan}\label{plan}}

  \begin{enumerate}
  \def\labelenumii{\arabic{enumii}.}
  \item ~
    \hypertarget{verification-and-validation-team}{%
    \subsection{Verification and Validation
    Team}\label{verification-and-validation-team}}
  \end{enumerate}
\end{enumerate}

\begin{longtable}[]{@{}
  >{\raggedright\arraybackslash}p{(\columnwidth - 4\tabcolsep) * \real{0.3261}}
  >{\raggedright\arraybackslash}p{(\columnwidth - 4\tabcolsep) * \real{0.2308}}
  >{\raggedright\arraybackslash}p{(\columnwidth - 4\tabcolsep) * \real{0.4431}}@{}}
\toprule()
\begin{minipage}[b]{\linewidth}\raggedright
\textbf{Member}
\end{minipage} & \begin{minipage}[b]{\linewidth}\raggedright
\textbf{Role}
\end{minipage} & \begin{minipage}[b]{\linewidth}\raggedright
\textbf{Responsibilities}
\end{minipage} \\
\midrule()
\endhead
Bo Liang & Project developer & Develop software \\
Dr. Brodbeck & reviewer & Review structure of the project \\
Ziyang Fang & Code reviewer & Performs static code analysis and unit
test coverage \\
\bottomrule()
\end{longtable}

\hypertarget{srs-verification-plan}{%
\subsection{SRS Verification Plan}\label{srs-verification-plan}}

\textbf{Method:}

• \textbf{Structured Review} (based on SRS Section 8 traceability
matrix).

\textbf{Steps:}

1.Invite neuroscience researchers to review the main function of the
tool.

2. Use a checklist to evaluate the main part of the document including

Completeness, Feasibility, Traceability

3. Reviewers will provide feedback through github Issue

4.Author will address all the issues and conduct a final review of the
document

\hypertarget{design-verification-plan}{%
\subsection{Design Verification Plan}\label{design-verification-plan}}

\textbf{Method:}

• Automated code test + Code Walkthrough + Design Document Review.

\textbf{Tools:}

• Doxygen for generating module dependency diagrams, combined with
design documents to clarify dependencies between different modules.

\hypertarget{implementation-verification-plan}{%
\subsection{Implementation Verification
Plan}\label{implementation-verification-plan}}

Unit Testing:

• pyTest will be used as the main test tool and some other visualization
libraries testing tool will be used too Such as compare\_ image function
in matplolib testing

• The Coverage goal ≥ 90\% (measured using pytest-cov).

\textbf{Static Analysis:}

• Use flake8 to check code syntax and clean it based on the instructions
provided

.• Use mypy as static type checker to checks type annotations to detect
type errors without running the code.

\hypertarget{automated-testing-and-verification-tools}{%
\subsection{Automated Testing and Verification
Tools}\label{automated-testing-and-verification-tools}}

\begin{longtable}[]{@{}
  >{\raggedright\arraybackslash}p{(\columnwidth - 2\tabcolsep) * \real{0.3557}}
  >{\raggedright\arraybackslash}p{(\columnwidth - 2\tabcolsep) * \real{0.6443}}@{}}
\toprule()
\begin{minipage}[b]{\linewidth}\raggedright
\textbf{Tool}
\end{minipage} & \begin{minipage}[b]{\linewidth}\raggedright
\textbf{Purpose}
\end{minipage} \\
\midrule()
\endhead
pytest & Functional and unit testing framework \\
flake8 & check code style \\
mypy & type annotation verification \\
Matplotlib & Image baseline comparison (image\_comparison) \\
GitHub Actions & Cross-platform CI/CD pipeline \\
\bottomrule()
\end{longtable}

\hypertarget{software-validation-plan}{%
\subsection{Software Validation Plan}\label{software-validation-plan}}

\textbf{Method:User Acceptance Testing (UAT)}.

\textbf{Steps:}

1. Demonstrate core functionality to external supervisors (Rev 0
presentation).

2. Collect user feedback (usability survey in the appendix).

\begin{enumerate}
\def\labelenumi{\arabic{enumi}.}
\setcounter{enumi}{3}
\item ~
  \hypertarget{system-tests}{%
  \section{System Tests}\label{system-tests}}

  \begin{enumerate}
  \def\labelenumii{\arabic{enumii}.}
  \item ~
    \hypertarget{tests-for-functional-requirements}{%
    \subsection{Tests for Functional
    Requirements}\label{tests-for-functional-requirements}}
  \end{enumerate}
\end{enumerate}

\textbf{Test Case1:}

\textbf{FR}:the interactive data plots generated by Live Neuro should be
exactly identical to original plots

\textbf{Input}: calculation result of TRF model

\textbf{Expected Output}: data plots generated by Live Neuro should be
exactly identical

to original plots

\textbf{Verification Method}:image\_comparison method in matplotlib
testing

How test will be performed: this test will be automated by pytest

\textbf{Test Case2:}

\textbf{FR}:the output of LiveNeuro should be interactive between
different plots

\textbf{Input}: Selecting data points across linked charts.

\textbf{Expected Output}: Other charts highlight corresponding points
simultaneously.

\textbf{Verification Method}: Use mplcursors to trigger events and
compare data in the Corresponding plot to the input data

How test will be performed: this test will be executed by pytest and
manually

\begin{enumerate}
\def\labelenumi{\arabic{enumi}.}
\setcounter{enumi}{1}
\item ~
  \hypertarget{tests-for-nonfunctional-requirements}{%
  \subsection{Tests for Nonfunctional
  Requirements}\label{tests-for-nonfunctional-requirements}}

  \begin{enumerate}
  \def\labelenumii{\arabic{enumii}.}
  \item
    \textbf{Area of Testing1}
  \end{enumerate}
\end{enumerate}

\textbf{Test Case3}: Usability

• \textbf{Category}: Non-functional, User-Centered, Manual Evaluation

• \textbf{Initial Condition}: The Live Neuro application is fully
operational and prepared for hands-on evaluation.

• \textbf{Test Inputs/Conditions}:

\begin{quote}
• Participants engage with the software in a simulated neuroscience
research workflow.

• A structured usability survey (Appendix 6.2) is conducted, covering
ease of interaction, visualization clarity, and feature intuitiveness.
\end{quote}

• \textbf{Expected Outcome}:

\begin{quote}
• Participants can efficiently navigate core functionalities with
minimal guidance.

• Survey responses highlight strengths and areas for enhancement.
\end{quote}

• \textbf{Execution Steps}:

\begin{quote}
1. Deploy Live Neuro to a group of neuroscience researchers and graduate
students.

2. Observe user behavior and gather feedback on usability pain points.

3. Analyze survey data, identify usability trends, and refine the user
interface accordingly.
\end{quote}

\textbf{Test Case4}: Maintainability

• \textbf{Category}: Non-Functional, Code Reliability, Automated
Validation

• \textbf{Initial Condition}: The project contains distinct
computational modules, each responsible for a separate functionality.

• \textbf{Test Inputs/Conditions}:

• Run an automated test coverage analysis using pytest-cov.

• \textbf{Expected Outcome}:

\begin{quote}
• The overall test coverage remains above a predefined threshold
(≥90\%).

• Any uncovered sections of code are identified and targeted for
additional tests.
\end{quote}

• \textbf{Execution Steps}:

\begin{quote}
1. Execute all existing test cases with coverage tracking enabled.

2. Generate a detailed report showing per-module and per-function
coverage statistics.

3. Identify weakly tested components and extend test coverage
accordingly.
\end{quote}

Test Case5: Portability

• \textbf{Category}: Non-Functional, Deployment Readiness, Manual \&
Automated Testing

• \textbf{Initial Condition}: The Live Neuro source code is
version-controlled and ready for multi-platform testing.

• \textbf{Test Inputs/Conditions}:

\begin{quote}
• Attempt to install, build, and execute the software on Linux (Ubuntu
22.04), Windows 11, and macOS Ventura.
\end{quote}

• \textbf{Expected Outcome}:

\begin{quote}
• The software compiles without errors and runs smoothly on all tested
platforms.

• Platform-specific discrepancies (if any) are documented and addressed.
\end{quote}

• \textbf{Execution Steps}:

\begin{quote}
1. Configure test environments across different operating systems.

2. Build and execute Live Neuro, ensuring its core functionalities
(e.g., data preprocessing, EEG/MEG visualization) behave consistently.

3. Log and analyze any platform-dependent issues, optimizing
cross-platform performance as needed.
\end{quote}

Test Case6: Reusability

• \textbf{Category}: Non-Functional, Code Design Evaluation, Manual
Review

• \textbf{Initial Condition}: Live Neuro's codebase consists of distinct
functional components (e.g., data handling, signal processing,
interactive visualization).

• \textbf{Test Inputs/Conditions}:

• Review module separation and cross-component dependencies.

• \textbf{Expected Outcome}:

\begin{quote}
• The software architecture follows clear separation of concerns, with
reusable components structured for extensibility.
\end{quote}

• \textbf{Execution Steps}:

\begin{quote}
1. Analyze the code structure, identifying modules that can be
independently reused or extended.

2. Assess whether core computational functions are generalizable for
broader neuroscience research applications.

3. Document reusability strengths and suggest refactoring where
necessary to improve modularity.
\end{quote}

\hypertarget{traceability-between-test-cases-and-requirements}{%
\subsection{Traceability Between Test Cases and
Requirements}\label{traceability-between-test-cases-and-requirements}}

\begin{longtable}[]{@{}
  >{\raggedright\arraybackslash}p{(\columnwidth - 2\tabcolsep) * \real{0.5000}}
  >{\raggedright\arraybackslash}p{(\columnwidth - 2\tabcolsep) * \real{0.5000}}@{}}
\toprule()
\begin{minipage}[b]{\linewidth}\raggedright
\begin{quote}
\textbf{Test Case}
\end{quote}
\end{minipage} & \begin{minipage}[b]{\linewidth}\raggedright
\begin{quote}
\textbf{Covered Requirements}
\end{quote}
\end{minipage} \\
\midrule()
\endhead
\begin{minipage}[t]{\linewidth}\raggedright
\begin{quote}
\textbf{Test Case1}
\end{quote}
\end{minipage} & \begin{minipage}[t]{\linewidth}\raggedright
\begin{quote}
R1
\end{quote}
\end{minipage} \\
\begin{minipage}[t]{\linewidth}\raggedright
\begin{quote}
\textbf{Test Case2}
\end{quote}
\end{minipage} & \begin{minipage}[t]{\linewidth}\raggedright
\begin{quote}
R2
\end{quote}
\end{minipage} \\
\begin{minipage}[t]{\linewidth}\raggedright
\begin{quote}
\textbf{Test Case3}
\end{quote}
\end{minipage} & \begin{minipage}[t]{\linewidth}\raggedright
\begin{quote}
NFR1
\end{quote}
\end{minipage} \\
\begin{minipage}[t]{\linewidth}\raggedright
\begin{quote}
\textbf{Test Case4}
\end{quote}
\end{minipage} & \begin{minipage}[t]{\linewidth}\raggedright
\begin{quote}
NFR2
\end{quote}
\end{minipage} \\
\begin{minipage}[t]{\linewidth}\raggedright
\begin{quote}
\textbf{Test Case5}
\end{quote}
\end{minipage} & \begin{minipage}[t]{\linewidth}\raggedright
\begin{quote}
NFR3
\end{quote}
\end{minipage} \\
\begin{minipage}[t]{\linewidth}\raggedright
\begin{quote}
\textbf{Test Case6}
\end{quote}
\end{minipage} & \begin{minipage}[t]{\linewidth}\raggedright
\begin{quote}
NFR4
\end{quote}
\end{minipage} \\
\bottomrule()
\end{longtable}

\begin{enumerate}
\def\labelenumi{\arabic{enumi}.}
\setcounter{enumi}{4}
\item ~
  \hypertarget{unit-test-description}{%
  \section{Unit Test Description}\label{unit-test-description}}

  \begin{enumerate}
  \def\labelenumii{\arabic{enumii}.}
  \item ~
    \hypertarget{unit-testing-scope}{%
    \subsection{Unit Testing Scope}\label{unit-testing-scope}}
  \end{enumerate}
\end{enumerate}

\textbf{Included Modules:}

• data visualization.

\textbf{Excluded Modules:}

• Native Matplotlib functions (assumed verified).

\hypertarget{tests-for-functional-requirements-1}{%
\subsection{Tests for Functional
Requirements}\label{tests-for-functional-requirements-1}}

\textbf{UT-F1:} \textbf{Interactive Visualization Sync}

\begin{quote}
• \textbf{Objective}: Ensure multi-chart linking updates correctly when
a user selects data points.

• \textbf{Test Type}: Functional, Automated

• \textbf{Test Input}: User selects a data point in one chart.

• \textbf{Expected Output}: The corresponding data point is highlighted
in all linked charts.
\end{quote}

\hypertarget{tests-for-nonfunctional-requirements-1}{%
\subsection{Tests for Nonfunctional
Requirements}\label{tests-for-nonfunctional-requirements-1}}

\textbf{UT-NF1: Usability -- Interface Clarity and User Experience}

\begin{quote}
• \textbf{Objective}: Ensure tooltips, labels, and messages are clear
and helpful.

• \textbf{Test Type}: Non-Functional, Manual Inspection

• \textbf{Test Input}: Verify UI components like tooltips, axis labels,
and error messages.

• \textbf{Expected Output}:

• All UI elements provide relevant and accurate feedback.

• User interactions result in expected system responses.
\end{quote}

\textbf{UT-NF2: Maintainability -- Code Readability and Structure}

\begin{quote}
• \textbf{Objective}: Ensure the codebase follows maintainable coding
practices.

• \textbf{Test Type}: Non-Functional, Static Analysis

• \textbf{Test Input}: Run Flake8 and Pylint on the project.

• \textbf{Expected Output}: No major code quality issues detected.
\end{quote}

\textbf{UT-NF3: Maintainability -- Test Coverage}

\begin{quote}
• \textbf{Objective}: Ensure each module has sufficient test coverage.

• \textbf{Test Type}: Non-Functional, Automated

• \textbf{Test Input}: Run pytest with coverage tracking.

• \textbf{Expected Output}: Test coverage is at least \textbf{90\%}.
\end{quote}

\textbf{UT-NF4: Portability -- Multi-Platform Compatibility}

\begin{quote}
• \textbf{Objective}: Verify the software runs on Linux, Windows, and
macOS.

• \textbf{Test Type}: Non-Functional, Automated

• \textbf{Test Input}: Detect OS type and verify execution.

• \textbf{Expected Output}: Live Neuro runs successfully on all
platforms.
\end{quote}

\textbf{UT-NF5: Reusability -- Module Independence}

\begin{quote}
• \textbf{Objective}: Ensure that major components are modular and
reusable in other projects.

• \textbf{Test Type}: Non-Functional, Manual Review

• \textbf{Test Input}: Check for modularity in different code files.

• \textbf{Expected Output}:

• Modules should not have unnecessary dependencies.

• Core functions should be easily extractable and reusable.
\end{quote}

\hypertarget{traceability-between-test-cases-and-modules}{%
\subsection{Traceability Between Test Cases and
Modules}\label{traceability-between-test-cases-and-modules}}

\begin{longtable}[]{@{}
  >{\raggedright\arraybackslash}p{(\columnwidth - 2\tabcolsep) * \real{0.5000}}
  >{\raggedright\arraybackslash}p{(\columnwidth - 2\tabcolsep) * \real{0.5000}}@{}}
\toprule()
\begin{minipage}[b]{\linewidth}\raggedright
\begin{quote}
\textbf{Test Case}
\end{quote}
\end{minipage} & \begin{minipage}[b]{\linewidth}\raggedright
\begin{quote}
\textbf{Covered Module}
\end{quote}
\end{minipage} \\
\midrule()
\endhead
\begin{minipage}[t]{\linewidth}\raggedright
\begin{quote}
\textbf{UT-F1}
\end{quote}
\end{minipage} & \begin{minipage}[t]{\linewidth}\raggedright
\begin{quote}
R1.R2
\end{quote}
\end{minipage} \\
\begin{minipage}[t]{\linewidth}\raggedright
\begin{quote}
\textbf{UT-NF1}
\end{quote}
\end{minipage} & \begin{minipage}[t]{\linewidth}\raggedright
\begin{quote}
NF1
\end{quote}
\end{minipage} \\
\begin{minipage}[t]{\linewidth}\raggedright
\begin{quote}
\textbf{UT-NF2}
\end{quote}
\end{minipage} & \begin{minipage}[t]{\linewidth}\raggedright
\begin{quote}
NF2
\end{quote}
\end{minipage} \\
\begin{minipage}[t]{\linewidth}\raggedright
\begin{quote}
\textbf{UT-NF3}
\end{quote}
\end{minipage} & \begin{minipage}[t]{\linewidth}\raggedright
\begin{quote}
NF2
\end{quote}
\end{minipage} \\
\begin{minipage}[t]{\linewidth}\raggedright
\begin{quote}
\textbf{UT-NF4}
\end{quote}
\end{minipage} & \begin{minipage}[t]{\linewidth}\raggedright
\begin{quote}
NF3
\end{quote}
\end{minipage} \\
\begin{minipage}[t]{\linewidth}\raggedright
\begin{quote}
\textbf{UT-NF5}
\end{quote}
\end{minipage} & \begin{minipage}[t]{\linewidth}\raggedright
\begin{quote}
NF4
\end{quote}
\end{minipage} \\
\bottomrule()
\end{longtable}

\hypertarget{usability-survey-questionnaire}{%
\section{Usability Survey
Questionnaire}\label{usability-survey-questionnaire}}

\textbf{Section 1: General Experience}

1. How easy was it to install and set up Live Neuro?

• ☐ \textbf{1} - Very difficult

• ☐ \textbf{2} - Somewhat difficult

• ☐ \textbf{3} - Neutral

• ☐ \textbf{4} - Somewhat easy

• ☐ \textbf{5} - Very easy

2. How intuitive was the user interface?

• ☐ \textbf{1} - Very confusing

• ☐ \textbf{2} - Somewhat confusing

• ☐ \textbf{3} - Neutral

• ☐ \textbf{4} - Somewhat intuitive

• ☐ \textbf{5} - Very intuitive

3. Were you able to complete basic tasks (e.g., loading data,
visualizing neural activity) without external help?

• ☐ \textbf{1} - No, I could not complete the tasks

• ☐ \textbf{2} - No, I needed significant assistance

• ☐ \textbf{3} - Yes, but I struggled

• ☐ \textbf{4} - Yes, but with minor difficulties

• ☐ \textbf{5} - Yes, without any difficulty

\textbf{Section 2: Visualization and Interaction}

4. How clear and informative were the generated EEG/MEG visualizations?

• ☐ \textbf{1} - Very unclear

• ☐ \textbf{2} - Somewhat unclear

• ☐ \textbf{3} - Neutral

• ☐ \textbf{4} - Somewhat clear

• ☐ \textbf{5} - Very clear

5. Were the interactive features (e.g., zooming, selecting data points,
multi-chart linking) responsive and easy to use?

• ☐ \textbf{1} - Very unresponsive and difficult to use

• ☐ \textbf{2} - Somewhat unresponsive

• ☐ \textbf{3} - Neutral

• ☐ \textbf{4} - Somewhat responsive and easy to use

• ☐ \textbf{5} - Very smooth and intuitive

6. Did the software provide sufficient feedback when interacting with
the visualizations (e.g., tooltips, highlights, error messages)?

• ☐ \textbf{1} - No feedback at all

• ☐ \textbf{2} - Some feedback, but often unclear

• ☐ \textbf{3} - Neutral

• ☐ \textbf{4} - Mostly clear feedback

• ☐ \textbf{5} - Very clear and helpful feedback

\textbf{Section 3: Performance and Reliability}

7. How would you rate the speed of data processing and visualization?

• ☐ \textbf{1} - Extremely slow, unusable

• ☐ \textbf{2} - Slow but functional

• ☐ \textbf{3} - Neutral

• ☐ \textbf{4} - Acceptable speed

• ☐ \textbf{5} - Very fast

8. Did you experience any crashes, errors, or unexpected behaviors?

• ☐ \textbf{1} - Very frequent crashes/issues

• ☐ \textbf{2} - Frequent minor issues

• ☐ \textbf{3} - Neutral

• ☐ \textbf{4} - Rare minor issues

• ☐ \textbf{5} - No issues at all

\textbf{Section 4: Cross-Platform Compatibility}

9. How well did Live Neuro perform on your operating system?

• ☐ \textbf{1} - Did not work at all

• ☐ \textbf{2} - Many issues, barely usable

• ☐ \textbf{3} - Neutral

• ☐ \textbf{4} - Mostly functional with minor issues

• ☐ \textbf{5} - Fully functional without problems

10. Did you encounter any platform-specific issues?

• ☐ \textbf{1} - Major problems, software unusable

• ☐ \textbf{2} - Significant issues but still usable

• ☐ \textbf{3} - Neutral

• ☐ \textbf{4} - Minor platform differences

• ☐ \textbf{5} - No platform-related issues

\textbf{Section 5: Documentation and Support}

11. How helpful was the provided documentation (e.g., user guide, API
reference)?

• ☐ \textbf{1} - Not helpful at all

• ☐ \textbf{2} - Somewhat unhelpful

• ☐ \textbf{3} - Neutral

• ☐ \textbf{4} - Somewhat helpful

• ☐ \textbf{5} - Very helpful

12. What is your level of confidence in using Live Neuro after reading
the documentation?

• ☐ \textbf{1} - Not confident at all

• ☐ \textbf{2} - Slightly confident

• ☐ \textbf{3} - Neutral

• ☐ \textbf{4} - Mostly confident

• ☐ \textbf{5} - Fully confident

\textbf{Section 6: Overall Satisfaction \& Suggestions}

13. How satisfied are you with Live Neuro overall?

• ☐ \textbf{1} - Very dissatisfied

• ☐ \textbf{2} - Somewhat dissatisfied

• ☐ \textbf{3} - Neutral

• ☐ \textbf{4} - Somewhat satisfied

• ☐ \textbf{5} - Very satisfied

14. Would you recommend Live Neuro to other neuroscience researchers?

• ☐ \textbf{1} - Definitely not

• ☐ \textbf{2} - Probably not

• ☐ \textbf{3} - Neutral

• ☐ \textbf{4} - Probably yes

• ☐ \textbf{5} - Definitely yes

15. What features or improvements would you like to see in future
versions? \emph{(Open-ended)}

• \textbf{Your response:}
\_\_\_\_\_\_\_\_\_\_\_\_\_\_\_\_\_\_\_\_\_\_\_\_

\end{document}
